% !TEX TS-program = pdflatex
%
% Created by João Lourenço on 2024-12-27.
% Copyright (c) by João Lourenço 2022
\documentclass[a4paper,11pt]{article}


%-------------------------------------------------------------
% Configuração do pacote "uninovaletterhead"
%-------------------------------------------------------------
% \usepackage{uninovaletterhead}  % defaults: times new roman font, black headings
% \usepackage[everypage,opensans,blue]{uninovaletterhead}  % With blue section headings
\usepackage[everypage,opensans]{uninovaletterhead}

\unvresearchgroup{Grupo de Investigação}{Research Group}
\unvphone{+351 218 724 500}




%-------------------------------------------------------------
% Resto do docuemnto LaTeX
%-------------------------------------------------------------
\usepackage[T1]{fontenc}
\usepackage{calligra}
\usepackage{courier}
\usepackage[portuguese]{babel}
\usepackage{xltabular}
\usepackage[colorlinks]{hyperref}
\hypersetup{allcolors=unvblue}
% \geometry{hmargin=2.0cm,vmargin=2.5cm}

\newcommand*{\thePackage}{\texttt{\uninovaletterheadname}}

\unvtitle{O pacote \LaTeX\ \thePackage}
\unvauthor{João M. Lourenço, Professor Associado @ NOVA FCT}
% \unvauthor{John Doe, Investigador Principal}      % descomentar para ver múltiplas assinaturas
% \unvauthor{Alice Doe, Biolseiro}                  % descomentar para ver múltiplas assinaturas
\unvauthorsigfont{\LARGE\color{unvblue}\calligra}
\unvpositionsigfont{\scriptsize}

\title{\theTitle}
\author{\theAuthor}
\date{\uninovadate\ \ (v.\ \uninovaversion)}

\hypersetup{
  pdftitle   = {\theTitle},
  pdfsubject = {\theDepartment $|$ UNINOVA},
  pdfauthor  = {\theAuthor}
}


% The document
\begin{document}

\maketitle

% \thispagestyle{empty}% Não imprimir número de página na primeira página

\begin{abstract}
    Este documento é simultaneamente um manual de instruções e um exemplo de como usar o pacote “\thePackage”.  Este pacote permite produzir documentos em \emph{papel timbrado} do \href{www.uninova.pt}{UNINOVA (Instituto de Desenvolvimento de Novas Tecnologias)}.
\end{abstract}


\section{Preâmbulo}

A versão mais recente deste pacote está disponível para download em:

\begin{center}
  \url{github.com/joaomlourenco/uninovaletterhead}
\end{center}

Se não tem a certeza se está a utilizar a versão mais recente, aproveite e \href{github.com/joaomlourenco/uninovaletterhead/archive/refs/heads/main.zip}{faça download} da última versão!   Já agora, se achar este pacote útil, \href{www.paypal.com/donate/?hosted_button_id=8WA8FRVMB78W8}{ofereça um café} ao meu alter-ego \emph{NOVAthesis} e na caixa de comentários diga que é para o pacote \thePackage! ;)


\section{Usar o pacote \thePackage}

Este pacote \thePackage é simples de usar.  Basta adicionar no preâmbulo do ficheiro fonte, i.e., depois do \verb!\docuemntclass{…}! e antes do \verb!\begin{document}!, o seguinte comando:

\begin{verbatim}
  \usepackage[opções]{uninovaletterhead}
\end{verbatim}

\noindent onde, por omissão, apenas a primeira página será timbrada.
As opções válidas são:\vspace{-1.5ex}

\bgroup
  \renewcommand{\arraystretch}{1.5}
  \begin{xltabular}{\textwidth}{lX}
    \texttt{everypage}  & Todas as páginas serão timbradas e não apenas a primeira.\\
    \texttt{opensans}   & Utilizar a fonte \emph{OpenSans}.\\
  \end{xltabular}
\egroup

\section{Configurar o pacote \thePackage}

Ainda no preâmbulo, deverá configurar o seu docuemnto com os seguintes comandos:

\bgroup
  \renewcommand{\arraystretch}{1.5}
  \begin{xltabular}{\textwidth}{lX}
    \verb+\unvresearchgroup+    & Este comando recebe dois argumentos, o primeiro com o nome do grupo de investigação em Português e o segundo com o nome do grupo de investigação em Inglês.  Estes nomes são convertidos para maiúsculas e apresentados no canto superior direito.  \textbf{\textsl{Se omitido nada será apresentado no canto superior direito.}}\\
    \verb+\unvtitle+            & Este comando recebe como argumento o título do documento.\\
    \verb+\unvauthor+           & Este comando recebe como argumento o nome do autor e, opcionalmente, o seu título/posição, a colocar na zona de assinatura.  \textbf{\textsl{Se omitido nada será apresentado.  Se usado múltiplas vezes, serão criadas múltiplas zonas de assinatura, uma para cada um dos autores.}}\\
    \verb+\unvauthorsigfont+    &  Configurar a fonte para escrever o(s) nome(s) do(s) autor(es).\\
    \verb+\unvpositionsigfont+  &  Configurar a fonte para escrever a(s) posições(s) do(s) autor(es).\\
  \end{xltabular}
\egroup

\section{Configurar e Apresentar a Zona de Assinatura}

Há também um comando para criar uma zona para colocar uma assinatura.  Este comando tem três argumentos, sendo que o primeiro (entre “[…]”) é opcional.

\begin{verbatim}
  \unvsignature[tamanho_da_linha]{posição}{afastamento}
\end{verbatim}

\noindent onde:

\medskip
\bgroup
  \renewcommand{\arraystretch}{1.5}
  \begin{xltabular}{\textwidth}{lX}
    \texttt{tamanho\_da\_linha}  & \emph{Opcional} — Indicação do tamanho da linha (em qualquer unidade válida no \LaTeX, por exemplo, \emph{6cm}, \emph{2in}, \emph{30pt}, etc).  \textbf{\textsl{Se omitido desenha uma linha ligeiramente mais larga que o nome/posição.}}\\
    \texttt{posição} & \emph{Obrigatório} — Indicação da localização da zona de assinatura.  Valores possíveis:\newline
    \begin{tabular}[t]{>{\slshape\bfseries}ll}
      l & zona de assinatura à esquerda;\\
      c & zona de assinatura centrada; e\\
      r & zona de assinatura à direita.\\
    \end{tabular}\\
    \texttt{afastamento}  & \emph{Obrigatório} — Indicação do espaço a deixar entre o final do texto e a zona de assinatura (em qualquer unidade válida no \LaTeX, por exemplo, \emph{3cm}, \emph{2.5in}, \emph{10ex}, etc).\\
  \end{xltabular}
\egroup

A zona de assinatura que termina este documento, localizada em baixo à direita, foi criada com:

\begin{verbatim}
  \unvauthor{João M. Lourenço, Professor Associado @ NOVA FCT}
  \unvauthorsigfont{\LARGE\calligra}
  \unvpositionsigfont{\scriptsize}
  ...
  \unvsignature{r}{1.5cm}
\end{verbatim}

\section{Um comentário final}

Este trabalho está licenciado ao abrigo da
Licença pública do projeto LaTeX (LPPL), versão 1.3c.
Para ver uma cópia desta licença, visite
\url{https://www.latex-project.org/lppl/lppl-1-3c/}



\unvsignature{r}{1.5cm}

\end{document}

